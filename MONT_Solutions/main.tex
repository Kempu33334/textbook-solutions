\documentclass{article}
\usepackage{kempu}

\newcommand{\lcm}{\operatorname{lcm}}
\newcommand{\noproblems}{\emph{No problems.}}

\title{EGMO Solutions}
\author{Kempu33334}
\date{July 2025}

\begin{document}

\maketitle

\tableofcontents

\newpage

\section{Fundementals of Number Theory}

\subsection{Divisibility}

\noproblems

\subsection{Divisibility Properties}

\begin{problem}[1.2.1]{}
Show that if $n > 1$ is an integer, $n \nmid 2n^2 + 3n + 1$.
\end{problem}
Assume there exists such an $n$. Then, subtracting $n(2n+3)$ from the RHS of the condition, we find that $n \nmid 1$, so $n = 1$ or $-1$, which is a contradiction. $\square$

\begin{problem}[1.2.2]{}
Let $a > b$ be natural numbers. Show that $a \nmid 2a + b$.
\end{problem}
Assume for the sake of contradiction there exists $a > b$ where $a \mid 2a+b$. Then, $a \mid b$, implying that $a \le b$, which is a contradiction. $\square$

\begin{problem}[1.2.3]{}
For $2$ fixed integers $x$, $y$, prove that \[x-y \mid x^n-y^n\] for any non-negative integer $n$.
\end{problem}
Clearly, the statement is equivalent to $x^n-y^n \pmod{x-y} \equiv 0$. However, we can write that \[x^n-y^n \equiv (x-(x-y))^n-y^n \equiv 0 \pmod{x-y}\] as required. $\square$

\subsection{Euclid’s Division Lemma}

\noproblems

\subsection{Primes}

\begin{problem}[1.4.1]{}
Find all positive integers $n$ for which $3n-4$, $4n-5$, and $5n-3$ are all prime numbers.
\end{problem}
In order for $5n-3$ to be prime, we must have $n$ even or $n = 1$. Hence, make the transformation $n = 2n'$. Then, $3n-4 \mapsto 6n'-4$, which can never be prime other than when $n = 2$. Trying both $n=1$ and $n=2$, we find that only $n = \boxed{2}$ works. $\square$

\begin{problem}[1.4.2]{}
If $p < q$ are two consecutive odd prime numbers, show that $p + q$ has at least $3$ prime factors (not necessarily distinct).
\end{problem}
Clearly, it cannot have zero or one prime factor. If it has two prime factors, then we can express \[p+q = rs\] for some primes $r$ and $s$. However, we know that one of these has to be $2$, hence WLOG assume it is $r$. Then, \[\dfrac{p+q}{2} = s\] which implies that there exists a prime between $p$ and $q$, which contradicts the fact that they are consecutive, as required. $\square$

\subsection{Looking at Numbers as Multisets}

\noproblems

\subsection{GCD and LCM}

\begin{problem}[1.6.1]{}
Prove that $\gcd(a, b) = a$ if and only if $a \mid b$.
\end{problem}
We start with the if direction. Clearly, if $a = 2^{a_1}3^{a_2}\dots$ and $b = 2^{b_1}3^{b_2}\dots$, then the divisibility condition implies $a_i \le b_i$ for all $i \ge 1$. Hence, \[\min(a_i, b-i) = a_i\] which proves the claim.

For the only if direction, we know that $\min(a_i, b_i) = a_i$ for any $i \ge 1$, implying that $a_i \le b_i$, which proves the desired result. $\square$

\begin{problem}[1.6.2]{}
If $p$ is a prime, prove that $\gcd(a, p) \in \{1, p\}$.
\end{problem}
Clearly, the only divisors of $p$ are $1$ and $p$. $\square$

\begin{problem}[1.6.3]{}
Let $a$, $b$ be relatively prime. Show that if $a \mid c$, $b \mid c$, then $ab \mid c$.
\end{problem}
This is clear since $ab = \gcd(a,b)\lcm(a,b) = \lcm(a,b) \mid c$. $\square$

\begin{problem}[1.6.4]{}
Prove that if $p$ is a prime with $p \mid ab$, then $p \mid a$ or $p \mid b$.
\end{problem}
Clearly, if $p \nmid a$ and $p \nmid b$, then $p \nmid ab$, which is a contradiction. $\square$

\subsection{Euclid’s Division Algorithm}

\begin{problem}[1.7.1]{}
Find $\gcd(120, 500)$ using the algorithm.
\end{problem}
We have that \[\gcd(120, 500) = \gcd(120, 20) = \boxed{20}\] as required. $\square$

\begin{problem}[1.7.2]{}
Show that $\gcd(4n + 3, 2n) \in \{1, 3\}$.
\end{problem}
We note that \[\gcd(4n+3, 2n) = \gcd(3, 2n)\] which implies the conclusion. $\square$

\begin{problem}[1.7.3]{}
Let $a$, $b$ be integers. We can write $a = bq + r$ for integers $q$, $r$ where $0 \le r < b$. Then our lemma states that \[\gcd(a, b) = \gcd(r, b).\] However, is $\lcm(a, b) = \lcm(r, b)$?
\end{problem}
No. If so, then multiplying the two, we have that \[ab = rb \implies a = r\] which cannot be true. $\square$

\subsection{B\'ezout's Theorem}

\begin{problem}[1.8.1]{}
Let $a$, $b$, $x$, $y$, $n$ be integers such that \[ax + by = n.\] Prove that $\gcd(a, b)$ divides $n$.
\end{problem}
Clearly, since $\gcd(a, b)$ divides the LHS, it must also divide the RHS, as required. $\square$

\begin{problem}[1.8.2]{}
Let $(a, b) = (8, 12)$. Find $x$, $y \in \mathbb{Z}$ such that \[ax + by = \gcd(a, b).\]
\end{problem}
It suffices to find $x$ and $y$ satisfying \[2x+3y=1\] and clearly, $(x, y) = \boxed{(2, -1)}$ works. $\square$

\begin{problem}[1.8.3]{}
Let $(a, b) = (7, 12)$. Find $x$, $y \in \mathbb{Z}$ such that \[ax + by = \gcd(a, b).\]
\end{problem}
We must find $x$ and $y$ where \[7x+12y = 1\] but clearly $(x, y) = (7, -4)$ works, so we are done. $\square$
\subsection{Base Systems}

\begin{problem}[1.9.1]{}
Find $37$ in base $5$. Find $69$ in base $2$.
\end{problem}
The former is $122_5$, and the latter is $1000101_2$. $\square$

\begin{problem}[1.9.2]{}
Show that any power of $2$ is of the form $100\dots0_2$.
\end{problem}
This is clear, since $2^n$ will be expressed as $1\underbrace{00\dots0}_{n\text{ times}}$.

\begin{problem}[1.9.3]{}
Prove in general that if $n = a_0 \times \ell^0 + \dots + a_k \times \ell^k$, then $k$ is such that $\ell^k \le n < \ell^{k+1}$ and $a_k$ is such that $a_k\ell^k \le n < (a_k + 1)\ell^k$.
\end{problem}
Clearly, since $a_k \ge 1$, we have that $\ell^k \le n$. In addition, since $a_{k+1} = 0$, we have the other bound. Now, for the latter statement, the lower bound is obvious. The upper bound can be shown by considering that $a_i < \ell$ for all $i$ and using the geometric series formula.

\begin{problem}[1.9.4]{}
Let $k = \floor{\log_\ell(n)}$. Show that $n$ has exactly $k+1$ digits in base $\ell$.
\end{problem}
Note that since \[\ell^k = \ell^{\floor{\log_\ell(n)}} \le n\] we know that $n$ has at least $k+1$ digits in base $\ell$. In addition, \[\ell^{k+1} = \ell^{\floor{\log_\ell(\ell n)}} > \ell^{\log_\ell(\ell n)-1} = n\] which shows that there are at most $k+1$ digits, as required. $\square$

\subsection{Extra Results as Problems}

\begin{problem}[1.10.1]{}
Prove that if $ab = cd$, then $a + b + c + d$ is not a prime number.
\end{problem}
Substitute $a = pq$, $b = rs$, $c = pr$, and $d = qs$. Then, \[a+b+c+d = pq+pr+qs+rs = (q+r)(p+s)\] so we are done. $\square$

\subsection{Example Problems}

\noproblems

\subsection{Practice Problems}

\begin{problem}[1.12.1]{}
Show that any composite number $n$ has a prime factor $\le \sqrt{n}$.
\end{problem}
Assume not. Then, since $n$ has at least two prime factors, consider any two of them, say $p$ and $q$. Since $pq \le n$, we know that $p$ and $q$ cannot both be greater than $\sqrt{n}$, so at least one of them is $\le \sqrt{n}$, contradiction. $\square$

\begin{problem}[1.12.2]{IMO 1959/1}
Prove that for any natural number $n$, the fraction \[\dfrac{21n + 4}{14n + 3}\] is irreducible.
\end{problem}
We have that \[\gcd(21n+4, 14n+3) = \gcd(7n+1, 14n+3) = \gcd(7n+1, 1) = 1\] so they are relatively prime, as required. $\square$

\begin{problem}[1.12.3]{}
Let $x$, $y$, $a$, $b$, $c$ be integers.
\begin{enumerate}
\item Prove that $2x + 3y$ is divisible by $17$ if and only if $9x + 5y$ is divisible by $17$.
\item If $4a + 5b - 3c$ is divisible by $19$, prove that $6a - 2b + 5c$ is also divisible by $19$.
\end{enumerate}
\end{problem}
We start with the first statement and the if direction. We have that $9x+5y \pmod{17} \equiv 0$. Multiplying by $4$, we have that $36x+20y \pmod{17} \equiv 2x + 3y \equiv 0$ as required. For the only if direction, we can multiply $2x+3y \pmod{17} \equiv 0$ by $13$. 

For the second part, we have that $4a+5b-3c \pmod{19} \equiv 0$, and multiplying by $11$ gives the desired result. $\square$

\begin{problem}[1.12.4]{}
Define the $n$th Fermat number $F_n$ by $F_n = 2^{2^n}+ 1$. Show that $\gcd(F_m, F_n) = 1$ for any $m \neq n$.
\end{problem}
Assume for the sake of contradiction there exist $m \neq n$ such that $\gcd(F_m, F_n) \neq 1$. Then, let $p$ be some prime dividing $F_m$. Then, \[2^{2^m}+1 \equiv 0 \pmod{p} \implies 2^{2^{m+1}} \equiv 1 \pmod{p}.\] Hence the order of $2 \pmod{p}$ is $2^{m+1}$. Similarly, if $p$ divides $F_n$, then we find that the order of $2 \pmod{p}$ is $2^{n+1}$. However, these two quantities can only be equal if $m = n$, which is a contradiction of the original statement. $\square$

\begin{problem}[1.12.5]{}
Prove that for each positive integer $n$, there is a positive integer $m$ such that each term of the infinite sequence $m + 1$, $m^m + 1$, $m^{m^m} + 1$, $\dots$ is divisible by $n$. 
\end{problem}
If $n$ is even, then take $m = n-1$. This clearly works since \[(n-1)^{(n-1)^{(n-1)^{\dots}}} \equiv (-1)^{(n-1)^{(n-1)^{\dots}}} \equiv -1 \pmod{n}.\] If $n$ is odd, then take $m = 2n-1$. Then, we have that \[(2n-1)^{(2n-1)^{(2n-1)^{\dots}}} \equiv (-1)^{(2n-1)^{(2n-1)^{\dots}}} \equiv -1 \pmod{n}\] as required. $\square$

\begin{problem}[1.12.6]{Romanian Mathematical Olympiad}
Let $a$, $b$ be positive integers such that there exists a prime $p$ with the property $\lcm(a, a + p) = \lcm(b, b + p)$. Prove that $a = b$.
\end{problem}
We have that \[\dfrac{a^2+ap}{\gcd(a, p)} = \dfrac{b^2+bp}{\gcd(b, p)} \implies \dfrac{\gcd(b, p)}{\gcd(a, p)} = \dfrac{b^2+bp}{a^2+ap}.\] We now case on the $v_p$ of the two variables.

If $v_p(a) = v_p(b) = 0$ or $v_p(a), v_p(b) \ge 1$, then we have that \[a^2+ap = b^2+bp \implies (a-b)(a+b+p) = 0.\] Hence, either $a = b$, or one of $a$ or $b$ is negative, which we cannot have. Hence, this case is done.

Now, if $v_p(a) = 0$ and $v_p(b) \ge 1$, then we have that \[p(a^2+ap) = b^2+bp \implies a^2 p + a p^2 - b^2 - b p = 0\] however this means that $p \mid b$, so substituting $b = kp$, we have that \[a^2+ap-k^2p-kp = 0\] which implies the same thing as the case above, so $a = k$ implying that $b = ap$. However, this means that \[p = \dfrac{b^2+bp}{a^2+ap} = \dfrac{a^2p^2+ap^2}{a^2+ap} \implies a^2+ap = a^2p+ap\] so $p = 1$, which doesn't work.

The case where $v_p(a) \ge 1$ and $v_p(b) = 0$ is similar.

Hence, exhausted all cases, we are done. $\square$

\begin{problem}[1.12.7]{St. Petersburg 1996}
 Find all positive integers $n$ such that \[3^{n-1} + 5^{n-1} \mid 3^n + 5^n.\]
\end{problem}
We have that \[3^{n-1}+5^{n-1} \mid 3\cdot 3^{n-1}+5\cdot 5^{n-1} \implies 5^{n-1}-3^{n-1} \pmod{5^{n-1}+3^{n-1}} \equiv 0.\] Hence, we must have that $5^{n-1} = 3^{n-1}$ so $n = \boxed{1}$.

\begin{problem}[1.12.8]{Russia 2001 Grade 11 Day 2/2}
Let $a$, $b$ be naturals such that $ab(a+b)$ is divisible by $a^2 + ab + b^2$. Show that $|a - b| > \sqrt[3]{ab}$.
\end{problem}
Let $\gcd(a, b) = d$, so that $a = dm$ and $b = dn$. Then, \[m^2+mn+n^2 \mid dmn(m+n)\] and $\gcd(m, n) = 1$. In addition, we make a claim.
\begin{claim*}
If $\gcd(m, n) = 1$, then $\gcd(m^2+mn+n^2, mn(m+n)) = 1$.
\end{claim*}
\begin{proof}
Let $p$ be a prime dividing $m$. Then, notice that it also divides $mn(m+n)$. Now, in order for $p$ to divide $m^2+mn+n^2$, it would have to divide $n^2$, but $m$ and $n$ don't share a common prime factor. Hence, we have the required conclusion.
\end{proof}
As a result, we know that $m^2+mn+n^2 \mid d$, so $m^2+mn+n^2 \le d$. Hence, \[|a-b|^3 \ge d^2\cdot d|m-n|^3 \ge d^2(m^2+mn+n^2) = a^2+ab+b^2 > ab\] so we are done. $\square$

\begin{problem}[1.12.9]{Germany}
Let $m$ and $n$ be two positive integers where $\gcd(m, n) = 1$. Prove that for every positive integer $k$, $n + m$ is a divisor of $n^2 + km^2$ if and only if $n + m$ is a divisor of $k + 1$.
\end{problem}
We start with the only if direction. Since \[n^2+km^2 \equiv m^2+km^2 \equiv (k+1)m^2 \equiv 0 \pmod{m+n}\] and as $\gcd(m, m+n) = 1$, we know that $m+n$ divides $k+1$.

We proceed with the if direction. Since \[0 \equiv k+1 \equiv (k+1)m^2 \equiv m^2+km^2 \equiv n^2+km^2 \pmod{m+n}\] as required. $\square$

\begin{problem}[1.12.10]{Japan 2020 Junior Finals P3}
Find all tuples of positive integers $(a, b, c)$ such that \[4\lcm(a, b, c) = ab + bc + ca.\]
\end{problem}
WLOG let $a \le b \le c$. Then, we know that $a \mid bc$, $b \mid ac$, and $c \mid ab$. Now, since $\lcm(a, b, c) \mid ab$, we may make the following claim.
\begin{claim*}
We claim that $\lcm(a, b, c) = ab$.
\end{claim*}
\begin{proof}
Clearly, if it is not equal to $ab$, then $\lcm(a, b, c) \le \tfrac{ab}{2}$. Then, substituting gives that \[ab = bc+ac\] which cannot work. Hence, we have the required conclusion.
\end{proof}
Hence, substituting $\lcm(a, b, c) = ab$, we find that \[3ab = bc+ac\] and from the claim above, $\gcd(a, b) = 1$. Hence, we find that either $c \mid 3$, $c \mid a$, or $c \mid b$. We now case.
\begin{itemize}
\item If $c \mid 3$, then either $c = 3$ or $c = 1$. If it is the latter, then we must have $a = b = c = 1$, which clearly does not work. If it is the former, then we wish to find solutions to $ab = a+b$ which factors as $(a-1)(b-1) = 1$, so we must have $a = b = 2$. Trying this, we see that this does not work.
\item If $c \mid a$, then we know that $a = b = c$, so trying this, \[4a = 3a^2 \implies a = \dfrac{4}{3}\] which does not work.
\item If $c \mid b$, then we know that $b = c$, so the equation reduces to \[4\lcm(a, b) = 2ab+b^2.\] Now, if $\lcm(a, b) = ab$, then we know that $2ab = b^2$ so $2a = b$, but then we must have $b = c = 2$, so $a = 1$. Trying this, we see that this is indeed a solution. Else, we know that $\lcm(a, b) = \tfrac{ab}{2}$, but this cannot lead to solutions.
\end{itemize}
Hence, the only solution is $\boxed{(1, 2, 2)}$ and permutations. $\square$

\newpage

\begin{problem}[1.12.11]{Iran MO 2017 Round 2/1}
Prove the following:
\begin{enumerate}
\item There doesn’t exist a sequence $a_1, a_2, a_3, \dots$ of positive integers such that for all $i < j$, we have $\gcd(a_i + j, a_j + i) = 1$.
\item Let $p$ be an odd prime number. Prove that there exists a sequence $a_1, a_2, a_3, \dots$ of positive integers such that for all $i < j$, $p \nmid \gcd(a_i + j, a_j + i)$.
\end{enumerate}
\end{problem}
We start with the first part. Notice that by selecting $i = 2m$ and $j = 2n$, we find that any even indexed term must be odd. Then, by taking $i = 2m$ and $j = 2n-1$, we find that all odd indexed terms must be odd also. However, selecting $i = 2m-1$ and $j = 2n-1$ then gives a contradiction since $2$ must divide it.

We finish with the second part. Notice that $\gcd(a_i + j, a_j + 1) = \gcd((a_i+i)+(a_j+j), a_j+1)$. Hence, it suffices to find a sequence $\{a\}$ where \[p \nmid (a_i+i)+(a_j+j).\] However, we can just select the sequence where $a_i = (i-1)p+2-i$, so that \[\{a\} = 1, p, 2p-1, 3p-2, \dots.\] Then, notice that the sum of any two terms must be $4 \pmod{p}$, as required. $\square$

\begin{problem}[1.12.12]{Russia 2017 Grade 10 Day 1/5}
Suppose $n$ is a composite positive integer. Let $1 < a_1 < a_2 < \dots < a_k < n$ be all the divisors of $n$. It is known, that $a_1 + 1, \dots , a_k + 1$ are all divisors for some $m$ (except $1$, $m$). Find all such $n$.
\end{problem}
We find that we must have \[(a_1+1)(a_k+1) = (a_2+1)(a_{k-1}+1) = \dots\] and as a result, \[a_1+a_k = a_2+a_{k-1} = \dots.\] Now, looking at the first equality, we must have that \[a_1+\dfrac{n}{a_1} = a_2+\dfrac{n}{a_2} \implies a_1-a_2 = n\left(\dfrac{1}{a_2}-\dfrac{1}{a_1}\right) = \dfrac{n(a_1-a_2)}{a_1a_2}.\] Hence, either $a_1 = a_2$ or $a_1a_2 = n$. Clearly, we cannot have the latter, so $\tau(n) \le 4$. We now case.
\begin{itemize}
\item If $\tau(n) = 2$, then it must be prime, which contradicts the problem statement.
\item If $\tau(n) = 3$, then $n = p^2$ for some prime $p$. Then, we require for $p+1$ to be the all the divisors of some $m$, but $2 \mid p+1$ unless $p = 2$, for which we find the solution $n = 4$.
\item If $\tau(n) = 4$, then either $n = p^3$ or $n = pq$ for primes $p$, $q$. If it is the former, then there must exist $m$ such that all the divisors of $m$ are $p+1$ and $p^2+1$. However, notice that unless $p = 2$, both are divisible by two, so this is impossible. If $p = 2$, then notice that $m = 15$ works, so this is satisfactory. On the other hand, if $n = pq$, then WLOG $p < q$. In that case, $p+1$ and $q+1$ must be all the divisors of $m$, however this is impossible for the same reason as the previous case.
\end{itemize}
Hence, in the end, our only solutions are $n = \boxed{4, 8}$, as required. $\square$

\begin{problem}[1.12.13]{IMO 2002/4}
Let $n \ge 2$ be a positive integer, with divisors $1 = d_1 < d_2 < \dots < d_k = n$. Prove that $d_1d_2 + d_2d_3 + \dots + d_{k-1}d_k$ is always less than $n^2$, and determine when it is a divisor of $n^2$.
\end{problem}
We first prove the first part. Notice that $\max(d_i) = \tfrac{n}{k-i+1}$. Hence, we wish to show that \[\dfrac{n}{1}\cdot \dfrac{n}{2}+\dfrac{n}{2}\cdot\dfrac{n}{3}+\dots+\dfrac{n}{k-1}\cdot\dfrac{n}{k} < n^2\] which is clear by telescoping.

We now show the second part. We start with a claim.
\begin{claim*}
In order for $d_1d_2 + d_2d_3 + \dots + d_{k-1}d_k \mid n^2$, $n$ cannot be composite.
\end{claim*}
\begin{proof}
Suppose for the sake of contradiction that there exists composite $n$ that works. Let $p$ be the smallest prime dividing $n$. Then, notice that \[d_1d_2 + d_2d_3 + \dots + d_{k-1}d_k > d_{k-1}d_k = \dfrac{n^2}{p}\] so it cannot work, as required.
\end{proof}
Trying primes $n = p$, we find that we must have $p+1 \mid p^2$, or \[\gcd(p^2, p+1) = p+1.\] Solving this, we require for $1 = p+1$, which has no solutions. Hence, there exist no solutions. $\square$

\begin{problem}[1.12.14]{Russia 2001 Grade 10 Day 2/4}
Find all odd positive integers $n > 1$ such that if $a$ and $b$ are relatively prime divisors of $n$, then $a + b - 1$ divides $n$.
\end{problem}
We claim the answer is $n = p^k$ for all odd primes $p$ and positive integers $k$, which clearly works.

We begin with a claim.
\begin{claim*}
We claim $n$ cannot have more than one prime factor.
\end{claim*}
\begin{proof}
Assume there exists, for the sake of contradiction, satisfactory $n$ which contains more than one prime factor. Then, we write $n = p^km$, where $p \nmid m$ and $m \neq 1$. In that case, notice that $p + m - 1$ must divide $n$. Now, since all the prime factors of $m$ are greater than $p$, we know that $\gcd(p-1, m) = 1$, so \[p+m-1 = p^{\ell}\] for some non-negative integer $\ell$. In addition, we know that $p^{\ell}+m-1 = 2p^{\ell}-p$ must divide $n$ and as a result, we know that $2p^{\ell-1}-1$ must divide $m$. As a result, \[2p^{\ell-1}-1 \mid p^{\ell}-p+1 \implies \gcd(2p^{\ell-1}-1, p^{\ell}-p+1) = \gcd(2p^{\ell-1}-1, p^{\ell-1}-(p-1)/2) = 2p^{\ell-1}-1.\] However, we must have that \[p^{\ell-1}+(p-1)/2 \le 1\] which is clearly impossible.
\end{proof}
As a result, we take $n = p^k$, which clearly works. $\square$

\begin{problem}[1.12.15]{INMO 2019/3}
Let $m$, $n$ be distinct positive integers. Prove that \[\gcd(m, n) + \gcd(m + 1, n + 1) + \gcd(m + 2, n + 2) \le 2|m - n| + 1.\] Further, determine when equality holds.
\end{problem}
WLOG let $m \ge n$, and notice that \[\gcd(m, n) + \gcd(m + 1, n + 1) + \gcd(m + 2, n + 2) = \gcd(m-n, n) + \gcd(m-n, n + 1) + \gcd(m-n, n + 2).\] Then, we know that at most one of $\gcd(m-n, n)$, $\gcd(m-n, n+1)$, and $\gcd(m-n, n+2)$ can be exactly $m-n$, while the others must be less than or equal to $\tfrac{m-n}{2}$ (as long as $m-n > 1$). As a result, we find that \[\gcd(m-n, n) + \gcd(m-n, n + 1) + \gcd(m-n, n + 2) \le m-n + \dfrac{m-n}{2} + \dfrac{m-n}{2} = 2(m-n) < 2(m-n)+1\] which is good. On the other hand, we have equality if and only if $m-n = 1$ or $m-n = 2$ with even $n$. $\square$

\begin{problem}[1.12.16]{USAMO 2007/1}
Let $n$ be a positive integer. Define a sequence by setting $a_1 = n$ and for each $k > 1$, letting $a_k$ to be the unique integer in the range $0 \le a_k \le k - 1$ for which $a_1 + a_2 + \cdots + a_k$ is divisible by $k$. For instance, when $n = 9$, the obtained sequence is $9$, $1$, $2$, $0$, $3$, $3$, $3$, $\dots$. Prove that for any $n$, the sequence $\{a\}$ eventually becomes constant.
\end{problem}

Notice that if there exists $k$, \[\dfrac{a_1+a_2+\cdots+a_k}{k} < k\] then the sequence will become constant for obvious reasons (simply selecting $\tfrac{a_1+a_2+\cdots+a_k}{k}$ for all successive elements suffices). Hence, assume that this does not happen. Then, we know that
\begin{align*}
a_1 &\ge 1 \\
a_1+a_2 &\ge 4 \\
&\vdots
\end{align*}
However, the sequence $\{a\}$ is bounded by $a_i \le i-1$, and since $n^2 - \tfrac{n(n+1)}{2}$ is monotonically increasing, this cannot hold. Hence, we have the required conclusion. $\square$

\begin{problem}[1.12.17]{USAMO 2007/5}
Prove that for every nonnegative integer $n$, the number $7^{7^n} + 1$ is the product of at least $2n + 3$ (not necessarily distinct) primes.
\end{problem}
We induct on $n$. For the base case, it is clearly true for $n = 0$. Now, assume it is true for some $n$, so that $m = 7^{7^n}$ is the product of at least $2n + 3$ primes. Then, \[7^{7^{n+1}}+1 = m^7 + 1 = (m+1)(m^6-m^5+m^4-m^3+m^2-m+1).\] We may write that \[m^6-m^5+m^4-m^3+m^2-m+1 = (m+1)^6 - 7m(m^2+m+1)^2\] which is a difference of squares. Hence, the total number of prime divisors is at least $2n+3 + 2 = 2n+5$, so we have the requireed conclusion.

\begin{problem}[1.12.18]{ELMO 2017/1}
Let $a_1, a_2, \dots, a_n$ be positive integers with product $P$, where $n$ is an odd positive integer. Prove that \[\gcd(a_1^n + P, a_2^n + P, \dots, a_n^n+ P) \le 2 \cdot \gcd(a_1, a_2, \dots , a_n)^n.\]
\end{problem}
Since the equation is homogeneous, assume that $\gcd(a_1, a_2, \dots, a_n) = 1$. Then, we wish to show that \[g = \gcd(a_1^n + P, a_2^n + P, \dots, a_n^n+ P) \le 2.\] Now, let $p \mid g$ be a prime. If there exists $i$ such that $p \mid a_i$, then we reach a contradiction with the condition on the $\gcd$ of all the $a$. Hence, $\gcd(g, P) = 1$. Now, since \[g \mid a_i^n+P\] over all $i$, we find that \[a_i^n \equiv -P \pmod{g}\] and a cyclic multiplication yields that \[2P^g \equiv 0 \pmod{g}\] so $g \mid 2$, as required. $\square$

\begin{problem}[1.12.19]{IMO 2001/6}
Let $a > b > c > d$ be positive integers and suppose that \[ac + bd = (b + d + a - c)(b + d - a + c).\] Prove that $ab + cd$ is not prime.
\end{problem}
We know that $b+d+a-c \mid ac+bd$, so \[\gcd(ac+bd, b+d+a-c) = \gcd((a+b)(a+d), b+d+a-c) = b+d+a-c\] so \[b+d+a-c \mid (a+b)(a+d).\] Similarly, we know that $b+d-a+c \mid ac+bd$, so \[\gcd(ac+bd, b+d-a+c) = \gcd((b+c)(c+d), b+d-a+c) = b+d-a+c\] so \[b+d-a+c \mid (b+c)(c+d).\] Hence, we find that \[ac+bd \mid (a+b)(a+d)(b+c)(c+d) = ((ac+bd)+(ad+bc))((ab+cd)+(ac+bd))\] implying that \[ac+bd \mid (ad+bc)(ab+cd).\] We know that \[ab+cd > ac+bc > ad+bc\] by the Rearrangment Inequality. Hence, if $ab+cd$ were to be prime, then $ac+bd \mid ad+bc$ which isn't possible, so we have the required conclusion. $\square$

\section{Modular Arithmetic Basics}

\subsection{Motivation}

\begin{problem}[2.1.1]{}
Show that $a + n \equiv a \pmod{n}$.
\end{problem}
This is clear, since $n \equiv 0 \pmod{n}$. $\square$

\begin{problem}[2.1.2]{}
Let $a$, $n$ be fixed integers. Show that the set of integers $b$ such that $b \equiv a \pmod{n}$ form an arithmetic progression. What is the common difference?
\end{problem}
Since they all leave the same remainder when divided by $n$, we know that they form an arithmetic progress with common difference $n$. $\square$

\begin{problem}[2.1.3]{}
Show that the set of integers $a$ such that $a \equiv 0 \pmod{n}$ is the set of multiples of $n$.
\end{problem}
Clearly, this is just all numbers which are divisible by $n$. $\square$

\subsection{Remainder Idea}

\noproblems

\subsection{Residue Classes}

\begin{problem}[2.3.1]{}
Guess why the above classes are called ``residue'' classes.
\end{problem}
Residue is just what is left behind. $\square$

\begin{problem}[2.3.2]{}
Show that the number of the classes modulo $n$ is exactly $n$.
\end{problem}
There is a clear bijection between the numbers $0$, $1$, $\dots$, $n-1$ and each residue class. $\square$

\subsection{Basic Properties}

\begin{problem}[2.4.1]{}
Show that $ab$ has remainder $rs \mod n$ by writing $a = nx+r$ and $b = ny+s$ and evaluating $ab$.
\end{problem}
We may write that \[ab \equiv (nx+r)(ny+s) \equiv n(\cdots)+rs \equiv rs \pmod{n}\] as required. $\square$

\begin{problem}[2.4.2]{}
Find the remainder when $2^{10}$ is divided by $10$.
\end{problem}
We know $2^{10} = 1024$, so the answer is $4$. $\square$

\begin{problem}[2.4.3]{}
Find $1002 \times 560 \pmod{7}$.
\end{problem}
Since $7 \mid 560$, the answer is $0$. $\square$

\begin{problem}[2.4.4]{}
Show that if $a \equiv b \pmod{n}$, then $ka \equiv kb \pmod{n}$ for any integer $k$.
\end{problem}
We are given that $a-b$ is a multiple of $n$, so $k(a-b)$ is too. $\square$

\begin{problem}[2.4.5]{}
Show that $a - b \mid a^n - b^n$ for any integer n.
\end{problem}
We write the following: \[a^n-b^n \equiv b^n-b^n \equiv 0\pmod{a-b}\] as required. $\square$

\begin{problem}[2.4.6]{}
If $p$ is an odd prime, and $a$, $b$ are coprime, show that \[\gcd\left(\dfrac{a^p + b^p}{a + b}, a + b\right) \in \{1, p\}.\]
\end{problem}
We may write that \[\dfrac{a^p+b^p}{a+b} \equiv a^{p-1}-a^{p-2}b+\dots+b^{p-1} \equiv pa^{p-1} \pmod{a+b}\] and since $\gcd(a, b) = 1$, we clearly have the necessary conclusion. $\square$

\begin{problem}[2.4.7]{}
Let $f$ be a polynomial with integer coefficients. Show that $a - b | f(a) - f(b)$ for any integers $a$, $b$ is the same as saying $f(a + d) \equiv f(a) \pmod{d}$.
\end{problem}
Let $b = a-d$. Then, we find that \[d \mid f(a)-f(a-d)\] which is the required conclusion. $\square$

\begin{problem}[2.4.8]{}
Show that $ka \equiv kb \pmod{n}$ implies $a \equiv b \pmod{n}$ if and only if $\gcd(k, n) = 1$.
\end{problem}
Clearly, an inverse $k^{-1}$ with respect to modulo $n$ exists if and only if $\gcd(k, n) = 1$, so both directions are trivial. $\square$

\subsection{Two Special Equal Sets}

\noproblems

\subsection{Fermat's Little Theorem}

\begin{problem}[2.6.1]{}
Show that $a^p \equiv a \pmod{p}$ holds in the case when $\gcd(a, p) \neq 1$.
\end{problem}
Notice that by simply multiplying $a^{p-1} \equiv 1 \pmod{p}$ by $a$, we take care of the case when $a \equiv 0\pmod{p}$ and extend it, as required. $\square$

\begin{problem}[2.6.2]{}
Let $a$, $b$ be integers and $p$ a prime. Show that $p$ divides $ab^p - a^pb$.
\end{problem}
By Fermat's Little Theorem, \[ab^p-a^pb \equiv ab-ab \equiv 0 \pmod{p}\] as required. $\square$

\begin{problem}[2.6.3]{}
Find \[2^{50} \pmod{7}.\]
\end{problem}
By FLT, \[2^{50} \equiv 2^2 \equiv 4 \pmod{7}\] as desired. $\square$

\subsection{Inverses}

\begin{problem}[2.7.1]{}
Show that inverses multiply like fractions.
\end{problem}
Follows since multiplication is commutative. $\square$

\begin{problem}[2.7.2]{}
Find the inverse of all $\{1, 2, 3, 4, 5, 6, 7, 8, 9, 10\}$ modulo $11$.
\end{problem}
This is just the set $\{1, 6, 4, 3, 9, 2, 8, 7, 5, 10\}$ where each number is mapped to the respective number in the other set. $\square$

\begin{problem}[2.7.3]{}
Show that $0$ does not have an inverse modulo $p$. What about $p$?
\end{problem}
Since $0$ times anything is $0$, it cannot equal $1$ at any point. Similarly, since $p$ is equivalent to $0$ in $\mathbb{F}_p$, we have the required conclusion. $\square$

\begin{problem}[2.7.4]{}
Prove that if $a \neq 0 \pmod{p}$, then \[a^{p-2} \equiv a^{-1} \pmod{p}.\]
\end{problem}
Follows by dividing FLT by $a$. $\square$

\begin{problem}[2.7.5]{}
Prove that the inverse of $a^n$ is the $n$th power of the inverse of $a$. That is, \[(a^{-1})^n \equiv (a^n)^{-1} \pmod{p}.\]

Using this, find the inverse of $256$ modulo $47$.
\end{problem}
The first part follows by exponent rules. The second, \[256^{-1} \equiv (2^8)^{-1} \equiv (2^{-1})^8 \equiv 24^8 \equiv 9 \pmod{47}\] as required. $\square$

\subsection{Simple Properties of Inverses and Wilson's Theorem}

\begin{problem}[2.8.1]{}
Prove that if $n$ is any natural satisfying $(n - 1)! \equiv -1 \pmod{n}$, then $n$ must be a prime.
\end{problem}
Follows by CRT and decomposition if $n$ is composite. $\square$

\begin{problem}[2.8.2]{}
Let $p$ be a prime. Show that the remainder when $(p - 1)!$ is divided by $p(p - 1)$ is $p - 1$.
\end{problem}
Follows by CRT and Wilson's Theorem. $\square$

\begin{problem}[2.8.3]{}
Let $n$ be an integer. Calculate \[\gcd(n! + 1,(n + 1)!).\]
\end{problem}
We split into cases based on if $n+1$ is prime.

If $n+1$ is prime, then it suffices to calculate $\gcd(n!+1, n+1)$. However, by Wilson's Theorem, this is just $n+1$, so we have the required answer.

If $n+1$ is composite, then let $q$ be a prime dividing $(n+1)!$. Then, we know that $q \le n$, so it divides $n!$ but not $n!+1$, so $\gcd(n!+1,(n+1)!) = 1$. $\square$

\subsection{General Equal Sets}

\noproblems

\subsection{Euler's Theorem}

\begin{problem}[2.10.1]{}
Find $2^{98} \pmod{33}$
\end{problem}
Since $\phi(33) = 20$, this reduces to \[2^{-2} \equiv 4^{-1} \equiv 25 \pmod{33}\] as required. $\square$

\begin{problem}[2.10.2]{}
Find $5^{30} \pmod{62}$.
\end{problem}
Since $\phi(62) = 30$, this is just $1 \pmod{62}$. $\square$

\begin{problem}[2.10.3]{}
What happens if $\gcd(a, n) \neq 1$? Does there exist any integer $m$ such that $a^m \neq 1 \pmod{n}$?
\end{problem}
No, since $a^m$ and $n$ will share a common factor. $\square$

\begin{problem}[2.10.4]{}
Show that $n \mid 2^{n!} - 1$ for all odd $n$.
\end{problem}
Notice that $\phi(n)$ will only have prime factors that are less than $n$, and $\phi(n) < n$, so we apply Euler's Theorem to win. $\square$

\subsection{General Inverses}

\begin{problem}[2.11.1]{}
Find the inverse of all $\{1, 3, 5, 7\}$ modulo $8$. What do you observe? Can you explain this?
\end{problem}
It is the set $\{1, 5, 3, 7\}$. It is just a permutation of the original set, but this is clear, since in order for it not to be, it would have to contain some number divisible by $2$, which cannot work. $\square$

\begin{problem}[2.11.2]{}
Does there exist an inverse for $5$ modulo $10$? What about $4$?
\end{problem}
No for both. $\square$

\begin{problem}[2.11.3]{}
Show that $\gcd(a^{-1}, n)$ is also $1$.
\end{problem}
Because $aa^{-1} \equiv 1 \pmod{n}$, if $\gcd(a^{-1}, n) \neq 1$, then they must share a common prime factor, which means that we cannot have $a$, contradiction. $\square$

\begin{problem}[2.11.4]{}
Prove that if $\gcd(a, n) \neq 1$, then a cannot have an inverse.
\end{problem}
Then, $a$ and $n$ must share a common prime factor, which means that there cannot exist an inverse. $\square$

\subsection{Extra Results as Problems}

\begin{problem}[2.12.1]{}
Use Freshman's dream and induction to prove Fermat's Little Theorem.
\end{problem}
We proceed using induction. For the base case, it is clearly true for $0$: \[0^p \equiv 0 \pmod{p}.\] Now, assume it is true for some $k$. Then, by Freshman's Dream, \[(k+1)^p \equiv k^p + 1 \equiv k+1 \pmod{p}\] as required. Hence, we are done. $\square$

\begin{problem}[2.12.2]{}
Use induction to show that \[(a + b)^{p^i} \equiv a^{p^i} + b^{p^i} \pmod{p}\] for any prime $p$ and any non-negative integer $i$.
\end{problem}
For the base case, it is clearly true for $i=0$. Hence, assume it is true for some $i$. Then, we will show that it is correct for $i+1$. Notice that \[(a+b)^{p^{i+1}} \equiv \left((a+b)^{p^i}\right)^p \equiv \left(a^{p^i}+b^{p^i}\right)^p \equiv a^{p+(i+1)}+b^{p+(i+1)} \pmod{p}\] as required. $\square$

\subsection{Example Problems}

\noproblems

\subsection{Practice Problems}

\begin{problem}[2.14.1]{}
How many prime numbers $p$ are there such that $29^p + 1$ is a multiple of $p$?
\end{problem}
We have that \[29^p + 1\equiv 30 \pmod{p}\] so the answer is $p = \boxed{2, 3, 5}$, which we can check to work. $\square$

\begin{problem}[2.14.2]{}
Let $p$ be a prime and $0 \le k \le p-1$ be an intger. Prove that \[\binom{p - 1}{k} \equiv (-1)^k \pmod{p}.\]
\end{problem}
We may write that \[\binom{p-1}{k} \equiv \dfrac{(p-1)!}{k!(p-1-k)!} \equiv \dfrac{-1}{k!(p-1-k)!} \equiv \dfrac{-1}{(-1)^k(p-1)!} \equiv (-1)^k \pmod{p}\] so we have the required conclusion. $\square$

\begin{problem}[2.14.3]{IMO 1979/1}
Let $a$ and $b$ be natural numbers such that \[\dfrac{a}{b} = 1 - \dfrac{1}{2} + \dfrac{1}{3} - \dfrac{1}{4} + \cdots - \dfrac{1}{1318} + \dfrac{1}{1319}.\] Prove that $a$ is divisible by $1979$. (Note: $1979$ is a prime)
\end{problem}
We know that \[\dfrac{a}{b} = \left(1+\dfrac{1}{2}+\cdots+\dfrac{1}{1319}\right) - \left(1+\dfrac{1}{2}+\cdots+\dfrac{1}{659}\right) = \dfrac{1}{660} + \dfrac{1}{661} + \cdots + \dfrac{1}{1319}.\] Now, by combining opposite fractions, each one is divisible by $1979$, so we have the required conclusion (and the number of fractions is even). $\square$

\begin{problem}[2.14.4]{RMO 2016/6}
Let $\{a\}$ be a strictly increasing sequence of positive integers in an arithmetic progression. Prove that there is an infinite subsequence of the given sequence whose terms are in a geometric progression.
\end{problem}
Notice that the terms of the form $a_1(d+1)^n$, where $d$ is the common difference, for some $n$ have the property that they are in the arithmetic sequence since by subtracting $a$, they are divisible by $d$, so we have the required conclusion. $\square$

\begin{problem}[2.14.5]{}
Let $f(x)$ be a polynomial with integer coefficients. Show that there does not exist a $N$ such that $f(x)$ is a prime for all $x \ge N$. In other words, $f(x)$ is not eventually always a prime. This problem shows that prime numbers don’t follow any polynomial pattern either.
\end{problem}
Notice that if $f(a)$ is prime, then $f(a+kp)$ must be composite infinitely many times, otherwise $f(x)$ is constant, so it suffices to pick $f(x) = p$ for a prime $p$. $\square$

\begin{problem}[2.14.6]{IMO 2005/4}
Determine all positive integers relatively prime to all the terms of the infinite sequence \[a_n = 2^n + 3^n + 6^n - 1, n \ge 1.\]
\end{problem}
Notice that for any prime $p$, selecting $n = p-2$ gives that \[2^{p-2}+3^{p-2}+6^{p-2}-1 \equiv \dfrac{1}{2}+\dfrac{1}{3}+\dfrac{1}{6}-1 \equiv 0 \pmod{p}\] so the only answer is $\boxed{1}$. $\square$

\begin{problem}[2.14.7]{IMO 1986/1}
Let $d$ be any positive integer not equal to $2$, $5$, or $13$. Show that one can find distinct $a$ and $b$ in the set $\{2, 5, 13, d\}$ such that $ab-1$ is not a perfect square.
\end{problem}

The problem is equivalent to showing that there exists $a \in \{2, 5, 13\}$ such that $ad-1$ is not a perfect square. Notice that if $\nu_2(d) \ge 1$, then taking $a = 2$ and modulo $4$ gives that it must be $3 \pmod{4}$, which is not possible, as required. Hence, assume $a$ is odd. Now, $5d-1$ is $2 \pmod{4}$ if $d \equiv 3 \pmod{4}$, so assume that $d \equiv 1 \pmod{4}$. Then, we write $d = 4k+1$, so that it is equivalent to $4ka+a-1$. However, substituting $a = 5$ and $a = 13$ respectively give that $5k+1$ and $13k+3$ are perfect squares, which is impossible by modulo $4$. Hence, one must not be a perfect square, and we are done. $\square$

\begin{problem}[2.14.8]{}
Let $a$ and $b$ be two relatively prime positive integers, and consider the arithmetic progression $a$, $a + b$, $a + 2b$, $a + 3b$, $\dots$.
\begin{enumerate}
\item Prove that there are infinitely many terms in the arithmetic progression that have the same prime divisors.
\item Prove that there are infinitely many pairwise relatively prime terms in the arithmetic progression.
\end{enumerate}
\end{problem}
We begin with the first part. Clearly, for any prime $p$, the terms that it divides is periodic modulo $p$, and it must be achieved at least once. Hence, we are done.

We finish with the second part. Suppose that there are finitely pany pairwise relatively prime terms in the arithmetic progression, and let them be $r_1$, $r_2$, $\dots$, $r_n$. Then, let $P = r_1r_2\cdots r_n$.
\begin{claim*}
There exists a non-negative integer $k$ such that for any prime $p \mid P$, $p \nmid a+kb$.
\end{claim*}
\begin{proof}
This is equivalent to \[kb \not\equiv -a \pmod{p}.\] Now, since $\gcd(b, p) = 1$, $b$ is invertible modulo $p$, so \[k \not\equiv -\dfrac{a}{b} \pmod{p}.\] Now, by CRT, we can construct such an $k$, so we are done.
\end{proof}
Hence, this contradicts the fact that we only have finitely many $r$, so we are done. $\square$

\begin{remark*}
The second part immediately follows from Dirichlet's Theorem on Arithmetic Progressions.
\end{remark*}

\begin{problem}[2.14.9]{}
Prove that:
\begin{enumerate}
\item Every positive integer has at least as many divisors of the form $4k + 1$ as divisors of the
form $4k + 3$.
\item There exist infinitely many positive integers which have as many divisors of the form
$4k + 1$ as divisors of the form $4k + 3$.
\item There exist infinitely many positive integers which have more divisors of the form $4k+1$
than divisors of the form $4k + 3$.
\end{enumerate}
\end{problem}

We begin with the first part and perform an strong induction. Clearly, the statement is true for $n = 1$. We now make a claim.
\begin{claim*}
If the statement is true for some $n$ and all $d$ dividing it, then it is true for any $np$ where $p$ is a prime.
\end{claim*}
\begin{proof}
We split into cases. If $p \nmid n$, then split into further cases.
\begin{itemize}
\item If $p \equiv 1 \pmod{4}$, then the result is clear.
\item If $p \equiv 3 \pmod{4}$, then the number of divisors that are $1 \pmod{4}$ and divide $n$ (call it $d_1$) is at least the number of divisors that are $3 \pmod{4}$ and divide $n$ (call it $d_3$). On the other hand, if the number of divisors that don't divide $n$ but do divide $pn$ and are congruent to $1 \pmod{4}$ is equal to $d_3$, while the number congruent to $3 \pmod{4}$ and divide $pn$ but not $n$ is $d_1$. Hence, the total amounts are equal.
\end{itemize}
Now, if $p \mid n$, then we can apply the same logic on $n = p^{\nu_p(n)+1}\cdot \tfrac{n}{p^{\nu_p(n)}}$ since $\tfrac{n}{p^{\nu_p(n)}} \mid n$ (by the strong induction). Hence, we are done.
\end{proof}

This then shows that all positive integers work.

We finish with the second and third part. Notice that the number $n = 3^{\ell}$ has $\left\lfloor\dfrac{1}{2}\ell\right\rfloor+1$ divisors congruent to $1 \pmod{4}$, while it has $\ell+1-\left\lfloor\dfrac{1}{2}\ell\right\rfloor$ divisors congruent to $3 \pmod{4}$. Now, these two are equal as long as $\ell$ is odd, and the first is greater than the second as long as $\ell$ is even, so we have the required conclusion. $\square$

\begin{remark*}
The second and third parts follow directly from Dirichlet's Theorem on Arithmetic Progressions on the sequences $a_i = 4i+1$ and $b_i = 4i+3$.
\end{remark*}

\begin{problem}[2.14.10]{Iberoamerican 2005/3}
Let $p \ge 5$ be a prime. Prove that if \[\sum_{i=1}^{p-1}\dfrac{1}{i^p} = \dfrac{m}{n}\] with $\gcd(m, n) = 1$, then $p^3 \mid m$.
\end{problem}

We know that \[\sum_{i=1}^{p-1}\dfrac{1}{i^p} = \sum_{i=1}^{\tfrac{p-1}{2}}\dfrac{i^p+(p-i)^p}{(i(p-i))^p}.\] We now make a claim.
\begin{claim*}
For any prime $p \ge 5$, \[i^p + (p-i)^p \pmod{p^2} \equiv 0.\]
\end{claim*}
\begin{proof}
Notice that by the Binomial Theorem, the coefficient of the $p^0$ term is $0$, and the coefficient of the $p$ term is $0$ when working in $\mathbb{F}_{p^2}$. Hence, we have the required conclusion.
\end{proof}
As a result, it suffices to show that \[\sum_{i=1}^{\tfrac{p-1}{2}}\dfrac{1}{(i(p-i))^p} \equiv \sum_{i=1}^{\tfrac{p-1}{2}}-i^{-2p}\equiv \sum_{i=1}^{\tfrac{p-1}{2}}-i^{-2} \equiv 0 \pmod{p}\] which is clear from the proof of Wolstenholme's Theorem (for a quick outline, each $i^{-2}$ maps to another distinct number by the properties of inverses, and we may sum all these normally). $\square$

\begin{problem}[2.14.11]{Sierpinski}
Prove that for any positive integer $s$, there is a positive integer $n$ whose sum of digits is $s$ and $s \mid n$.
\end{problem}

Notice that it suffices to find a sequence $\{a\}$ of non-negative distinct integers such that \[\sum_{i=1}^{s} 10^{a_i} \equiv 0 \pmod{s}\] since one can just add sufficiently many zeros to the end to take care of the factors of $2$ and $5$. However we can select $a_i$ such that $\phi(s) \mid a_i$ over all $i$, and then we are done. $\square$

\begin{problem}[2.14.12]{ISL 2001/N4}
Let $p \ge 5$ be a prime number. Prove that there exists an integer $a$ with $1 \le a \le p - 2$ such that neither $a^{p-1} - 1$ nor $(a + 1)^{p-1} - 1$ is divisible by $p^2$.
\end{problem}

Notice that there are exactly $p-1$ numbers $1 \le a \le p^2$ such that \[a^{p-1} \equiv 1 \pmod{p^2}\] so let these make the set $\mathcal{A}$. In addition, let $m$ be the largest integer such that $a_m \le p-1$. Then, if the given claim is false, then in each pair $\{1, 2\}$, $\{3, 4\}$, $\dots$, $\{p-2, p-1\}$, at least one of the numbers must be in $\mathcal{A}$. As a result, notice that $m \ge \tfrac{p-1}{2}$.

\begin{problem}[2.14.13]{USAMO 2018/4}
Let $p$ be a prime, and let $a_1$, $\dots$ , $a_p$ be integers. Show that there exists an integer $k$ such that the numbers \[a_1 + k, a_2 + 2k, \dots , a_p + pk\] produce at least $\tfrac{1}{2}p$ distinct remainders upon division by $p$.
\end{problem}

\begin{problem}[2.14.14]{Balkan 2016/3}
Find all monic polynomials $f$ with integer coefficients satisfying the following condition: there exists a positive integer $N$ such that $p$ divides $2(f(p)!) + 1$ for every prime $p > N$ for which $f(p)$ is a positive integer. (A monic polynomial has a leading coefficient equal to $1$.)
\end{problem}

\begin{problem}[2.14.15]{Iran MO 2017 Round 3/Final/NT/1}
Let $x$ and $y$ be integers and let $p$ be a prime number. Suppose that there exist relatively prime positive integers $m$ and $n$ such that \[x^m \equiv y^n \pmod{p}.\] Prove that there exists an unique integer $z$ modulo $p$ such that \[x \equiv z^n \pmod{p} \quad \text{and} \quad y \equiv z^m \pmod{p}.\]
\end{problem}

\begin{problem}[2.14.16]{ISL 2015/N3}
Let $m$ and $n$ be positive integers such that $m > n$. Define \[x_k = \dfrac{m + k}{n + k}\] for $k = 1$, $2$, $\dots$ , $n + 1$. Prove that if all the numbers $x_1$, $x_2$, $\dots$, $x_{n+1}$ are integers, then $x_1x_2\cdots x_{n+1} - 1$ is divisible by an odd prime.
\end{problem}

\begin{problem}[2.14.17]{ELMO 2019/5}
Let $\mathcal{S}$ be a nonempty set of positive integers such that, for any (not necessarily distinct) integers $a$ and $b$ in $\mathcal{S}$, the number $ab + 1$ is also in $\mathcal{S}$. Show that the set of primes that do not divide any element of $\mathcal{S}$ is finite.
\end{problem}
\end{document}
