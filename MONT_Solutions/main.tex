\documentclass{article}
\usepackage{kempu}

\title{EGMO Solutions}
\author{Kempu33334}
\date{July 2025}

\begin{asydef}
size(8cm); // set a reasonable default
usepackage("amsmath");
usepackage("amssymb");
settings.tex="pdflatex";
settings.outformat="pdf";
import geometry;
void filldraw(picture pic = currentpicture, conic g, pen fillpen=defaultpen, pen drawpen=defaultpen) { filldraw(pic, (path) g, fillpen, drawpen); }
void fill(picture pic = currentpicture, conic g, pen p=defaultpen) { filldraw(pic, (path) g, p); }
pair foot(pair P, pair A, pair B) { return foot(triangle(A,B,P).VC); }
pair centroid(pair A, pair B, pair C) { return (A+B+C)/3; }
\end{asydef}

\begin{document}

\maketitle

\tableofcontents

\newpage

\section{Fundementals of Number Theory}

\subsection{Divisibility}

\emph{No problems.}

\subsection{Divisibility Properties}

\begin{problem}[1.2.1]{}
Show that if $n > 1$ is an integer, $n \nmid 2n^2 + 3n + 1$.
\end{problem}
Assume there exists such an $n$. Then, subtracting $n(2n+3)$ from the RHS of the condition, we find that $n \nmid 1$, so $n = 1$ or $-1$, which is a contradiction. $\square$

\begin{problem}[1.2.2]{}
Let $a > b$ be natural numbers. Show that $a \nmid 2a + b$.
\end{problem}
Assume for the sake of contradiction there exists $a > b$ where $a \mid 2a+b$. Then, $a \mid b$, implying that $a \le b$, which is a contradiction. $\square$

\begin{problem}[1.2.3]{}
For $2$ fixed integers $x$, $y$, prove that \[x-y \mid x^n-y^n\] for any non-negative integer $n$.
\end{problem}
Clearly, the statement is equivalent to $x^n-y^n \pmod{x-y} \equiv 0$. However, we can write that \[x^n-y^n \equiv (x-(x-y))^n-y^n \equiv 0 \pmod{x-y}\] as required. $\square$

\subsection{Euclid’s Division Lemma}

\emph{No problems.}

\subsection{Primes}

\begin{problem}[1.4.1]{}
Find all positive integers $n$ for which $3n-4$, $4n-5$, and $5n-3$ are all prime numbers.
\end{problem}
In order for $5n-3$ to be prime, we must have $n$ even or $n = 1$. Hence, make the transformation $n = 2n'$. Then, $3n-4 \mapsto 6n'-4$, which can never be prime other than when $n = 2$. Trying both $n=1$ and $n=2$, we find that only $n = \boxed{2}$ works. $\square$

\begin{problem}[1.4.2]{}
If $p < q$ are two consecutive odd prime numbers, show that $p + q$ has at least $3$ prime factors (not necessarily distinct).
\end{problem}
Clearly, it cannot have zero or one prime factor. If it has two prime factors, then we can express \[p+q = rs\] for some primes $r$ and $s$. However, we know that one of these has to be $2$, hence WLOG assume it is $r$. Then, \[\dfrac{p+q}{2} = s\] which implies that there exists a prime between $p$ and $q$, which contradicts the fact that they are consecutive, as required. $\square$

\subsection{Looking at Numbers as Multisets}

\emph{No problems.}

\subsection{GCD and LCM}

\begin{problem}[1.6.1]{}
Prove that $\gcd(a, b) = a$ if and only if $a \mid b$.
\end{problem}
We start with the if direction. Clearly, if $a = 2^{a_1}3^{a_2}\dots$ and $b = 2^{b_1}3^{b_2}\dots$, then the divisibility condition implies $a_i \le b_i$ for all $i \ge 1$. Hence, \[\min(a_i, b-i) = a_i\] which proves the claim.

For the only if direction, we know that $\min(a_i, b_i) = a_i$ for any $i \ge 1$, implying that $a_i \le b_i$, which proves the desired result. $\square$

\begin{problem}[1.6.2]{}
If $p$ is a prime, prove that $\gcd(a, p) \in \{1, p\}$.
\end{problem}
Clearly, the only divisors of $p$ are $1$ and $p$. $\square$

\begin{problem}[1.6.3]{}
Let $a$, $b$ be relatively prime. Show that if $a \mid c$, $b \mid c$, then $ab \mid c$.
\end{problem}
This is clear since $ab = \gcd(a,b)\lcm(a,b) = \lcm(a,b) \mid c$. $\square$

\begin{problem}[1.6.4]{}
Prove that if $p$ is a prime with $p \mid ab$, then $p \mid a$ or $p \mid b$.
\end{problem}
Clearly, if $p \nmid a$ and $p \nmid b$, then $p \nmid ab$, which is a contradiction. $\square$

\subsection{Euclid’s Division Algorithm}

\begin{problem}[1.7.1]{}
Find $\gcd(120, 500)$ using the algorithm.
\end{problem}
We have that \[\gcd(120, 500) = \gcd(120, 20) = \boxed{20}.\] $\square$

\begin{problem}[1.7.2]{}
Show that $\gcd(4n + 3, 2n) \in \{1, 3\}$.
\end{problem}
We note that \[\gcd(4n+3, 2n) = \gcd(3, 2n)\] which implies the conclusion. $\square$

\begin{problem}[1.7.3]{}
Let $a$, $b$ be integers. We can write $a = bq + r$ for integers $q$, $r$ where $0 \le r < b$. Then our lemma states that \[\gcd(a, b) = \gcd(r, b).\] However, is $\lcm(a, b) = \lcm(r, b)$?
\end{problem}
No. If so, then multiplying the two, we have that \[ab = rb \implies a = r\] which cannot be true. $\square$

\subsection{B\'ezout's Theorem}

\emph{No problems.}

\subsection{Base Systems}

\begin{problem}[1.9.1]{}
Find $37$ in base $5$. Find $69$ in base $2$.
\end{problem}
The former is $122_5$, and the latter is $1000101_2$. $\square$

\begin{problem}[1.9.2]{}
Show that any power of $2$ is of the form $100\dots0_2$.
\end{problem}
This is clear, since $2^n$ will be expressed as $1\underbrace{00\dots0}_{n\text{ times}}$.

\begin{problem}[1.9.3]{}
Prove in general that if $n = a_0 \times \ell^0 + \dots + a_k \times \ell^k$, then $k$ is such that $\ell^k \le n < \ell^{k+1}$ and $a_k$ is such that $a_k\ell^k \le n < (a_k + 1)\ell^k$.
\end{problem}
Clearly, since $a_k \ge 1$, we have that $\ell^k \le n$. In addition, since $a_{k+1} = 0$, we have the other bound. Now, for the latter statement, the lower bound is obvious. The upper bound can be shown by considering that $a_i < \ell$ for all $i$ and using the geometric series formula.

\begin{problem}[1.9.4]{}
Let $k = \floor{\log_\ell(n)}$. Show that $n$ has exactly $k+1$ digits in base $\ell$.
\end{problem}
Note that since \[\ell^k = \ell^{\floor{\log_\ell(n)}} \le n\] we know that $n$ has at least $k+1$ digits in base $\ell$. In addition, \[\ell^{k+1} = \ell^{\floor{\log_\ell(\ell n)}} > \ell^{\log_\ell(\ell n)-1} = n\] which shows that there are at most $k+1$ digits, as required. $\square$

\subsection{Extra Results as Problems}

\begin{problem}[1.10.1]{}
Prove that if $ab = cd$, then $a + b + c + d$ is not a prime number.
\end{problem}
Substitute $a = pq$, $b = rs$, $c = pr$, and $d = qs$. Then, \[a+b+c+d = pq+pr+qs+rs = (q+r)(p+s)\] so we are done. $\square$

\subsection{Example Problems}

\emph{No problems.}

\subsection{Practice Problems}

\begin{problem}[1.12.1]{}
Show that any composite number $n$ has a prime factor $\le \sqrt{n}$.
\end{problem}
Assume not. Then, since $n$ has at least two prime factors, consider any two of them, say $p$ and $q$. Since $pq \le n$, we know that $p$ and $q$ cannot both be greater than $\sqrt{n}$, so at least one of them is $\le \sqrt{n}$, contradiction. $\square$

\newpage

\begin{problem}[1.12.2]{IMO 1959/1}
Prove that for any natural number $n$, the fraction \[\dfrac{21n + 4}{14n + 3}\] is irreducible.
\end{problem}
We have that \[\gcd(21n+4, 14n+3) = \gcd(7n+1, 14n+3) = \gcd(7n+1, 1) = 1\] so they are relatively prime, as required. $\square$

\begin{problem}[1.12.3]{}
Let $x$, $y$, $a$, $b$, $c$ be integers.
\begin{enumerate}
\item Prove that $2x + 3y$ is divisible by $17$ if and only if $9x + 5y$ is divisible by $17$.
\item If $4a + 5b - 3c$ is divisible by $19$, prove that $6a - 2b + 5c$ is also divisible by $19$.
\end{enumerate}
\end{problem}
We start with the first statement and the if direction. We have that $9x+5y \pmod{17} \equiv 0$. Multiplying by $4$, we have that $36x+20y \pmod{17} \equiv 2x + 3y \equiv 0$ as required. For the only if direction, we can multiply $2x+3y \pmod{17} \equiv 0$ by $13$. 

For the second part, we have that $4a+5b-3c \pmod{19} \equiv 0$, and multiplying by $11$ gives the desired result. $\square$

\begin{problem}[1.12.4]{}
Define the $n$th Fermat number $F_n$ by $F_n = 2^{2^n}+ 1$. Show that $\gcd(F_m, F_n) = 1$ for any $m \neq n$.
\end{problem}
Assume for the sake of contradiction there exist $m \neq n$ such that $\gcd(F_m, F_n) \neq 1$. Then, let $p$ be some prime dividing $F_m$. Then, \[2^{2^m}+1 \equiv 0 \pmod{p} \implies 2^{2^{m+1}} \equiv 1 \pmod{p}.\] Hence the order of $2 \pmod{p}$ is $2^{m+1}$. Similarly, if $p$ divides $F_n$, then we find that the order of $2 \pmod{p}$ is $2^{n+1}$. However, these two quantities can only be equal if $m = n$, which is a contradiction of the original statement. $\square$

\begin{problem}[1.12.5]{}
Prove that for each positive integer $n$, there is a positive integer $m$ such that each term of the infinite sequence $m + 1$, $m^m + 1$, $m^{m^m} + 1$, $\dots$ is divisible by $n$. 
\end{problem}
If $n$ is even, then take $m = n-1$. This clearly works since \[(n-1)^{(n-1)^{(n-1)^{\dots}}} \equiv (-1)^{(n-1)^{(n-1)^{\dots}}} \equiv -1 \pmod{n}.\] If $n$ is odd, then take $m = 2n-1$. Then, we have that \[(2n-1)^{(2n-1)^{(2n-1)^{\dots}}} \equiv (-1)^{(2n-1)^{(2n-1)^{\dots}}} \equiv -1 \pmod{n}\] as required. $\square$

\begin{problem}[1.12.6]{Romanian Mathematical Olympiad}
Let $a$, $b$ be positive integers such that there exists a prime $p$ with the property $\lcm(a, a + p) = \lcm(b, b + p)$. Prove that $a = b$.
\end{problem}
We have that \[\dfrac{a^2+ap}{\gcd(a, p)} = \dfrac{b^2+bp}{\gcd(b, p)} \implies \dfrac{\gcd(b, p)}{\gcd(a, p)} = \dfrac{b^2+bp}{a^2+ap}.\] We now case on the $v_p$ of the two variables.

If $v_p(a) = v_p(b) = 0$ or $v_p(a), v_p(b) \ge 1$, then we have that \[a^2+ap = b^2+bp \implies (a-b)(a+b+p) = 0.\] Hence, either $a = b$, or one of $a$ or $b$ is negative, which we cannot have. Hence, this case is done.

Now, if $v_p(a) = 0$ and $v_p(b) \ge 1$, then we have that \[p(a^2+ap) = b^2+bp \implies a^2 p + a p^2 - b^2 - b p = 0\] however this means that $p \mid b$, so substituting $b = kp$, we have that \[a^2+ap-k^2p-kp = 0\] which implies the same thing as the case above, so $a = k$ implying that $b = ap$. However, this means that \[p = \dfrac{b^2+bp}{a^2+ap} = \dfrac{a^2p^2+ap^2}{a^2+ap} \implies a^2+ap = a^2p+ap\] so $p = 1$, which doesn't work.

The case where $v_p(a) \ge 1$ and $v_p(b) = 0$ is similar.

Hence, exhausted all cases, we are done. $\square$

\begin{problem}[1.12.7]{St. Petersburg 1996}
 Find all positive integers $n$ such that \[3^{n-1} + 5^{n-1} \mid 3^n + 5^n.\]
\end{problem}
We have that \[3^{n-1}+5^{n-1} \mid 3\cdot 3^{n-1}+5\cdot 5^{n-1} \implies 5^{n-1}-3^{n-1} \pmod{5^{n-1}+3^{n-1}} \equiv 0.\] Hence, we must have that $5^{n-1} = 3^{n-1}$ so $n = \boxed{1}$.
\end{document}